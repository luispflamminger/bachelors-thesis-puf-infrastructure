\newpage
\section{Conclusion}
\label{sec:conclusion}


The thesis was able to lay a broad theoretical foundation about concepts surrounding PUFs, their types,
their implementations and PUF-based authentication.
The systematic review of the existing literature on PUF-based authentication protocols
provided an overview of the wide variety of applications, focuses, security properties
and performance factors. Through the literature review, it became clear, that the central part of any
infrastructure is the protocol, as it defines and limits the technical requirements of the underlying
hardware, backend systems and databases. Initially, it was planned to develop a new protocol fitting
the requirements of the research project. However, this goal was not realized, because the review
showed the complexity and knowledge required for designing a secure protocol goes far beyond
the scope of this thesis. A protocol fitting the requirements of the project was selected,
analyzed and a prototype was implemented.

When analyzing the methodology of the thesis critically, a couple of possible weak points can be identified.
Small parts of the fundamentals section were only based on few sources, which leaves a possibility of
incorrect and unverified information. However, the scientific quality and relevance of sources was always
ensured and no contradicting information was encountered, making this risk small.
When looking at the final selection of the protocol, it is possible, that the limited knowledge about
cryptography and security has lead to a false evaluation of certain protocols.
Additionally, the defined search string could have excluded relevant sources, opening the possibility,
that the best protocol was not found. Looking at the implementation, the security aspects
of the prototype were not closely examined.

To conclude the thesis, another look at the research questions defined in
section \ref{sec:research_questions} is necessary to evaluate the success of the thesis.
\emph{RQ1} is only partially answered by the thesis. Results almost exclusively focused on the protocol.
Other requirements for a full infrastructure, like the database implementation, the hardware required
for supporting RFID, or the requirements of access control systems, were only partially considered.

\emph{RQ2} required the implementation of a prototype and a demonstration of its
functionality. This goal was achieved as the selected protocol was implemented using
python and evaluated using a set of tests.
However, the prototype has its limitations. It has not been extensively tested
to ensure resilience against all possible types of attacks. Additionally, hardware constraints,
like a limited number of logic gates on RFID tags and the imperfect PUF responses in the real world,
have not been considered.