\newpage
\section{Review of Existing PUF-based Authentication Protocols}
\label{sec:review}

\subsection{Methodology for the Literature Review}
\label{sec:review_methodology}

To be able to conceptualize a protocol and the surrounding infrastructure, a review of the existing literature on
\ac{PUF}-based challenge-response authentication protocols is conducted.
In \cite{Brocke2009}, \citeauthor*{Brocke2009} propose guidelines for conducting literature reviews
in the information systems field. A summary of the five-step framework can be found in \ref{sec:appendix_brocke}.

Steps one and two of the framework are not necessary in the context of a bachelor's thesis, as the scope has been
defined in section \ref{sec:introduction} and fundamentals have been laid out extensively in section \ref{sec:fundamentals}.
During the research of fundamental topics, it became clear, that defining the authentication protocol is central to answering \emph{RQ1}.
Therefore, the following search string was developed:

\begin{quote}
    \emph{(physical unclonable function OR puf OR physically unclonable function) AND (authentication OR challenge response OR challenge-response) AND (protocol)}
\end{quote}

The databases all have slightly different syntax requirements for search strings, so the actual strings used
differ a bit from this one. The exact strings are shown in \ref{sec:appendix_search_strings}.

The search string was applied to IEEEXplore, ACM and Google Scholar and restricted to document titles.
IEEExplore yielded 53 results. Most of them are centered around lightweight PUF-based authentication protocols for \ac{IoT} applications.
Results were ordered by number of citations and evaluated by their titles and abstracts.
Some articles were deemed irrelevant and thus not considered. An example for this are \cite{Zhang2021}
and \cite{Yu2022},as they propose PUF-based authentication for distributed systems using blockchain technology,
which is out of scope for this thesis.

Next the same search was conducted in the ACM Digital Library. This yielded only three results, none of which were relevant.

To fill in the gaps, the same search was conducted using Google Scholar. This yielded 104 results, most of which were
irrelevant or already contained in the IEEExplore search results.
From this search, only articles that were well-cited, very recent or offered interesting new perspectives were considered.
This resulted in eight results across different publishers, namely Elsevier, Springer and MDPI.
Backward and forward search was not used to further increase the number of papers, as the sample size was
already large enough for the purposes of this review.

\subsection{Reviewed Articles}
\label{sec:review_collected_articles}

Table \ref{tbl:review_summary} shows an overview of the papers used, source database, number of citations and year published.
As analyzing all articles in depth would be too time-consuming, a subset of the most relevant results was selected beforehand.
During this process, articles with a higher number of citations were preferred. Additionally,
because this thesis is part of a research project, concerned with creating a physical access protection system using PUF-based authentication,
literature proposed for similar purposes was preferred.
Lastly, it was important that a wide variety of approaches was included. To ensure this,
if two papers had a similar approach, the one with more citations was preferred.

\begin{table}[H]
    \centering
    \caption{Summary of reviewed papers}
    \label{tbl:review_summary}
    \begin{tabular}{|l|c|c|l|c|}
        \hline
        \textbf{Article}                                         & \textbf{Citations} & \textbf{Year} & \textbf{Database} & \textbf{Section}            \\
        \hline\hline
        \citeauthor*{Majzoobi2012} \cite{Majzoobi2012}           & 124                & 2012          & IEEExplore        & \ref{sec:review_protocol_1} \\
        \hline\citeauthor*{Gope2018} \cite{Gope2018}             & 103                & 2018          & IEEExplore        & \ref{sec:review_protocol_2} \\
        \hline\citeauthor*{Chatterjee2019} \cite{Chatterjee2019} & 95                 & 2019          & IEEExplore        & \ref{sec:review_protocol_3} \\
        \hline\citeauthor*{Braeken2018} \cite{Braeken2018}       & 78                 & 2018          & Google Scholar    & \ref{sec:review_protocol_4} \\
        \hline\citeauthor*{Bansal2020} \cite{Bansal2020}         & 48                 & 2020          & IEEExplore        & \ref{sec:review_protocol_5} \\
        \hline\citeauthor*{Zhu2019} \cite{Zhu2019}               & 19                 & 2019          & Google Scholar    & \ref{sec:review_protocol_6} \\
        \hline\citeauthor*{Jiang2019} \cite{Jiang2019}           & 10                 & 2019          & IEEExplore        & \ref{sec:review_protocol_7} \\
        \hline\citeauthor*{Gope2022} \cite{Gope2022}             & 9                  & 2022          & IEEExplore        & \ref{sec:review_protocol_8} \\
        \hline\citeauthor*{Hristea2019} \cite{Hristea2019}       & 4                  & 2019          & Google Scholar    & \ref{sec:review_protocol_9} \\
        \hline
    \end{tabular}
\end{table}

\subsection{Analysis of the collected Literature}
\label{sec:review_analysis}

Next, each of the papers is analyzed.
The focus of analysis was put on a couple of central questions:
\begin{itemize}
    \item What is the goal of the protocol and what problem does it solve?
    \item What requirements to the underlying infrastructure (hardware, databases) does the protocol have?
    \item What kinds of attacks does the protocol aim to prevent?
    \item Is the protocol a good solution for the research questions posed in section \ref{sec:research_questions}
\end{itemize}

\subsubsection{Slender PUF Protocol by \citeauthor*{Majzoobi2012}}
\label{sec:review_protocol_1}

This protocol requires the use of a PUF with a specific statistical property called the
strict PUF avalanche criterion.
It describes PUFs, which behave in a way, that any bit-flip in the challenge will result in half of
the response bits to flip. To achieve this, the protocol uses a PUF consisting of eight individual
Arbiter PUFs connected through XOR logic gates, which improves the statistical properties to fulfill
this criterion. \cite{Majzoobi2012}

As a strong PUF, it has an exponentially large set of CRPs, which limits attacks by recording previously
used \acp{CRP} (replay attacks), because the same challenge does not have to be used multiple times.
This method is also used in the basic authentication protocol discussed in section \ref{sec:basic_challenge_response_protocol}, where
CRPs are deleted from the database after usage.
This safeguard alone does not protect against machine learning attacks, where an untrusted third-party
can use previously recorded CRPs to train a model of the PUF. This model learns the response behavior
and potentially allows an attacker to calculate the appropriate response to a future challenge.
This protocol aims to make this kind of attack impossible \cite[][p. 34f]{Majzoobi2012}.

The protocol assumes that the verifier knows the relationship of challenges and responses for each
instance through a compact model of the PUF, which has been trained during the enrollment phase
and is stored in a database on the verifiers side. The model can be trained with as many CRPs as necessary,
because the PUF can be directly evaluated. After enrollment, the physical access points
that allow direct evaluation, need to be disabled to prevent potential attackers with physical access
from evaluating the PUF directly and and building a model. This is done by blowing a fuse to disable the
external pins used for access to the PUF. After this, the PUF can only be accessed through a controller on the
device, which adheres to the protocol specifications and prevents direct readouts.
The only other way to obtain knowledge about the challenge-response behavior
at this point would be by opening the PUF, which would likely change its characteristics. \cite{Majzoobi2012}

In the paper, three types of attacks on this protocol are discussed.
The first is a PUF modeling attack, where an attacker trains a model on previously
recorded CRPs to try and imitate a PUF instance. The basic PUF authentication protocol
does not protect against this kind of attack, as the full challenges and responses are
exposed during the authentication process. This protocol mitigates this risk, by not directly
exposing responses. Instead, only substrings with a random starting index are exposed.
There is still a way for an attacker to create a model by guessing the substring each time,
but the amount of CRPs required to successfully build a model increases dramatically
which greatly reduces the risk. Of course, this attack vector would have to be analyzed
much more closely before deploying a system using this protocol. \cite{Majzoobi2012}

The second is a random guessing attack.
The paper gives concrete figures for the probability of such an attack being successful.
For example, a protocol implementation with a response length of 1024 bits, substring length of 256
bits and Hamming authentication threshold of 76 would only give a $10^-11$ chance of a successful guess. \cite{Majzoobi2012}

The paper also discuss attacks that compromise the randomness of the seed by impersonating
either verifier or prover, but these are deemed to be fixable depending on the implementation. \cite{Majzoobi2012}

The main benefits of this protocol include the resiliency against ML attacks, the robustness
against noise in PUF responses without the need for traditional expensive error correction. \cite{Majzoobi2012}

\subsubsection{Lightweight Anonymous Protocol for RFID Systems by \citeauthor*{Gope2018}}
\label{sec:review_protocol_2}

This protocol was developed for using PUFs as a cryptographic primitive for
entity authentication in IoT using RFID technology.
A typical RFID system involves an RFID tag, a reader and a backend server.
The protocol requires the tag to consist of a micro controller and a tamper-proof PUF with
intrinsic random variations. It also assumes a secure connection between the reader and the backend. \cite{Gope2018}

There are many security issues and attack vectors with traditional RFID systems.
One problem with typical RFID tags is, that they threaten the user's privacy,
because identifying information stored on the tag can be retrieved without immediate
access to the device and without need for authentication by the other party.
This allows tracking the location of users.
Attacks against forward secrecy are another problem. As the communication channel in RFID
is insecure, an eavesdropper can record all traffic, although it is usually encrypted.
If at some point in the future, the encryption is broken, all previously recorded communication
can now be accessed by the attacker.
Another problem is a desynchronization attack, where a third party blocks messages between
reader and tag, to desynchronize communication. In many classical RFID authentication protocols,
the secret is exchanged frequently to provide forward secrecy. If an attacker blocks the acknowledgement
from the server, that the secret was successfully changed, the tag does not know which secret to use.
This will cause a denial of service for that tag.
Additional attacks include impersonation attacks, physical attacks, where information is retrieved
by reading the physical memory on the tag, or cloning attacks, where a copy of the tag is created. \cite{Gope2018}

This protocol aims to protect against these attacks. For example, PUFs can introduce anonymity and protect the privacy of users.
Mutual authentication between server and tag is used to protect against impersonation attacks and location tracking.
The researchers analyzed many previously proposed authentication schemes and improved upon them.
Additionally, the protocol exhibits resilience against \ac{DoS} attacks and
forward secrecy. It does not require secret keys to be stored on the tag and does not require
computationally expensive operation on the backend side, which makes it scalable. \cite{Gope2018}

\subsubsection{Protocol without explicit CRPs in Verifier Database by \citeauthor*{Chatterjee2019}}
\label{sec:review_protocol_3}

This protocol focuses on a PUF-based authentication solution for a network of distributed IoT
devices, that does not require CRPs to be stored in a database
on the verifier's side. This is important, because in all of the previously
described protocols, exposure of the database is a high security risk.
The use of \ac{IBE} and keyed hash functions allows this protocol to
only require the verifier to store a single key in its storage \cite[][p. 1f]{Chatterjee2019}.
\ac{IBE} is a type of public key encryption, where the public key can be any string \cite[][p. 8]{Chatterjee2011}.
This differs from regular \ac{PKI}, where the keys have to have a specific length and
format, dictated by the encryption system (e.g. AES). \cite[][p. 3]{Chatterjee2011}

Protecting that single key is much easier than ensuring security for an entire database.
The protocol is designed, so that a prover does not have to store a key and does not need
to do expensive computational work. The protocol allows for an unlimited number of
authentications using the same prover \cite[][p. 2f]{Chatterjee2019}.
The paper mathematically proves, that the protocol is resistant against eavesdropping attacks. \cite[][p. 8]{Chatterjee2019}

Requirements include, that each prover has to have a PUF module with intrinsic random variations
and some additional mathematical and cryptographic modules for scalar multiplication,
pairing operations and cryptographic hashing.
The verifier is required to be able to calculate a keyed hash function.  \cite[][p. 1f]{Chatterjee2019}


% If needed, this protocol can easily be described, as prosa description of steps is given on page 4


\subsubsection{Protocol for IoT by \citeauthor*{Braeken2018}}
\label{sec:review_protocol_4}

This protocol was developed for IoT applications. It aims to solve problems with a previously
published protocol and is thus developed to protect against \ac{MITM}, \ac{DoS},
impersonation and replay attacks. Additionally it offers mutual authentication, which means
that both the node and the server prove their identity to each other.
The protocol focuses on power efficiency and low computational cost \cite[][p. 1f, 8]{Braeken2018}.
It relies on storing a large amount of CRPs in a database on the verifier side and
requires a strong PUF \cite[][p. 4]{Braeken2018}.
In addition to simple authentication, it also allows the communication of two nodes
with each other, even though a central verification server is still needed. \cite[][p. 3f]{Braeken2018}

During enrollment, CRPs are shared between prover and verifier and stored in the database.
After authentication, a public and private key pair are generated. Public keys can be shared
with other nodes, with the server acting as a trusted party facilitating the exchange. \cite[][p. 4]{Braeken2018}


\subsubsection{Lightweight Protocol for V2G by \citeauthor*{Bansal2020}}
\label{sec:review_protocol_5}

This protocol is also developed for IoT applications. More specifically, it is designed to be used
in a vehicle smart grid ecosystem, which would allow to tightly integrate electric cars into the power grid
and use their batteries for energy trading and load management, enabling a more efficient use of the grid's
energy. This approach is also called \ac{V2G} and would help to manage the ever increasing power demand world
wide. However, the communication between the grid and vehicles is a privacy and security concern, because
a lot of sensitive information is exchanged between the parties, which an attacker could compromise. This
could lead to imbalanced energy transactions or privacy violations. Physical access to devices is also
easy, because vehicles are usually parked for a long time without supervision.
PUFs are used here as a means of tamper-proofing and protection against physical attacks, as no keys
need to be stored. Tampering likely changes the characteristic of the PUF and cloning is impossible. \cite[][p. 7234]{Bansal2020}

The system consists of three parties, the vehicle, an aggregator (charging station) and the power grid.
Both the aggregators and the vehicles have PUF modules, as both entities need to be authenticated
with privacy and security.
Mutual authentication is required between grid and aggregator, as well as vehicle and aggregator, to ensure
a secure exchange of encryption keys, which can be used for further communication.
To protect privacy, the keys are generated using the PUF during each authentication run and pseudonyms are used
for identifying vehicles to protect their information. \cite[][p. 7234f]{Bansal2020}
Only one challenge response pair needs to be stored on the grid server for every vehicle.
It achieves this with significantly reduced computational overhead compared to similar protocols for V2G in the field. \cite[][p. 7244]{Bansal2020}

There are four central security goals of the protocol.
Messages need to be confidential, so only authorized entities can access the information.
The integrity of messages needs to be ensured, so aggregators and the grid can know, if a message has been altered or
compromised in any way. The identity of vehicle owners has to be kept hidden from an attacker even when eavesdropping.
Mutual authentication is also a requirement to prevent impersonation attacks and false energy transfer \cite[][p. 7237]{Bansal2020}.
The paper formally proves resiliency against man-in-the-middle attacks, impersonation attacks, replay attacks and
physical attacks. It is additionally able to provide mutual authentication, protection of identity, message integrity
and security of session keys. \cite[][p. 7243]{Bansal2020}

\subsubsection{Lightweight mutual RFID protocol by \citeauthor*{Zhu2019}}
\label{sec:review_protocol_6}

This protocol is another attempt at creating a mutual authentication protocol for RFID systems.
It focuses on the same problems and goals as \cite{Gope2018}, which were discussed
in section \ref{sec:review_protocol_2}, and also makes the same assumptions about the infrastructure \cite[][p. 1, 5]{Zhu2019}.
Because the paper was published after \cite{Gope2018}, it identifies some weaknesses of that protocol and
many other proposed implementations in the RFID space and tries to improve on them.
Specifically, it identifies an issue with the protection against desynchronization attacks,
where a tag can run out of pseudo-identities that are used for authentication, if communication
is denied by an attacker too many times. This leads to the tag needing to be re-enrolled.
This protocol is able to correct that flaw. \cite[][p. 4]{Zhu2019}
Another drawback to the protocol in \cite{Gope2018} is, that it requires active RFID tags, which have an
internal power source, whereas this protocol is able to be implemented with passive tags. \cite[][p. 18]{Zhu2019}

The paper provides proofs, that the protocol is secure against a large number of attacks, including
universal untraceability of tags, ensurance of forward secrecy, secure mutual authentication,
resilience against desynchronization attacks, resiliency against physical and cloning attacks and immunity
to modeling attacks \cite[][p. 12-15]{Zhu2019}.
All of these attacks are very relevant in an RFID setting.
It is also able to do this more efficiently both in space complexity and bandwidth requirements and
comes close to other protocols in computational complexity.
The protocol is considered to be scalable. \cite[][p. 16]{Zhu2019}

\subsubsection{Two-Factor protocol by \citeauthor*{Jiang2019}}
\label{sec:review_protocol_7}

A new approach is introduced here, which has not been discussed in any other paper considered here.
This protocol combines PUF-based authentication with a password that the user enters,
to create a \ac{2FA} authentication protocol.
It is designed to be used in IoT, specifically \ac{IoV}, for managing vehicles and
automating intelligent route planing. \cite[][p. 195]{Jiang2019}

As with \cite{Bansal2020}, PUFs are a solid foundation for security in vehicles,
because they are good at preventing physical attacks, which is a big issue
with cars that are left unattended for long periods of time.
The combination with a password ensures, that a vehicle can only be authenticated if the owner
or a different trusted party is present. \cite[][p. 195]{Jiang2019}

The architecture of the protocol involves three parties, a data center which manages the information, users which
receive data from sensors and statistical data used for route planning from the data center, and the vehicle
sensors which collect real-time traffic data. It involves three phases, including system
setup, registration and login/authentication \cite[][p. 197]{Jiang2019}.
The protocol needs to be able to protect data including the vehicles
location or driving data, like speed and fuel consumption, to ensure privacy and traffic safety.
It also needs to protect against attackers injecting false information about external conditions,
threatening the safety of passengers. \cite[][p. 195f]{Jiang2019}

The paper proves that the protocol successfully implements mutual authentication, providing anonymity.
Vehicles are not identifiable or traceable, as no easily identifiable information is sent.
The protocol is secure against physical attacks because of the PUF-based nature.
The paper also claims that the protocol is secure against impersonation and desynchronization attacks,
but does not specifically argue that point.
Only one CRP per entity has to be stored by the server, which is another security benefit. \cite[][p. 199f]{Jiang2019}


\subsubsection{Protocol level approach to prevent ML attacks by \citeauthor*{Gope2022}}
\label{sec:review_protocol_8}

This protocol is aimed at IoT systems in the medical field.
The main focus is to provide a protocol which protects against machine learning attacks on PUFs.
The architecture of the systems includes several medical IoT devices like wearable health sensors or
clinical systems, which communicate with a central IoT gateway. This gateway talks to a service provider
over the internet, who provides the authentication server, information database and analysis systems.
As with many of the other papers, the wireless nature of \ac{IoMT} devices opens the system up to a
number of attacks, such as replay, man-in-the-middle, impersonation, physical and guessing attacks.
Additionally modeling attacks are a big threat. \cite[][p. 1971f]{Gope2022}

The proposed protocol aims to improve upon several previously proposed protocols including \cite{Majzoobi2012},
which is part of this review. \citeauthor*{Gope2022} criticize it, because it does not provide sufficient protection
against replay and impersonation attacks \cite[][p. 1972]{Gope2022}.
This protocol uses a special type of PUF, called \ac{OPUF}, which changes its behavior after each session,
making modelling entirely impractical.
Benefits of the protocol offer mutual authentication preserving privacy, prevention of modelling attacks,
good scalability, prevention of man-in-the-middle and replay attacks \cite[][p. 1972]{Gope2022}. It also provides forward
secrecy and prevents physical attacks. These security characteristics are argued and verified through security
simulation software \cite[][p. 1979]{Gope2022}.

\subsubsection{Destructive private mutual RFID protocol by \citeauthor*{Hristea2019}}
\label{sec:review_protocol_9}

% References 19 and 13 in this paper are concerned with comprehensive security analysis of RFID, could be interesting
% for seperate section
This protocol is another attempt at a mutual authentication protocol for RFID systems \cite[][p. 331]{Hristea2019}.
It assumes the same architecture as \cite{Zhu2019} and \cite{Gope2018}, consisting of a tag, reader and
server component. The main threats this protocol is trying to solve are impersonation and tracking
of tags. The main benefit this paper offers, is that it is stateful, meaning that it allows a
tag to have persistent state, which can be updated and shared with the reader, while also providing
resilience against physical attacks, mutual authentication and forward secrecy. \cite[][p. 332]{Hristea2019}
It is also scalable and able to prevent desynchronization attacks.
The security of the protocol is proven using a privacy model for RFID protocols. \cite[][p. 340f]{Hristea2019}

% Uses PUFs to prevent physical attacks revealing the internal secret
% Destructive privacy, no identification possible by untrusted parties due to mutual authentication

\subsection{Discussion of Results}
\label{sec:review_results}

In this section, the results of the literature review are discussed.

The methodology used for collecting the relevant literature to be reviewed is described in detail in section \ref{sec:review_methodology}.
The database queries with the respective search strings lead to a set of papers which were mostly very recent.
Almost all papers were published between 2018 and 2022, with one outlier, \cite{Majzoobi2012}, published in 2012.
An important criteria for the selection of papers was also the number of citations, as this was used as a measure of relevance
in the field. Having a lot of citations does not prove that the paper is well received, as it could also be used as a "negative example"
by other researchers. This possibility was not further investigated, and might be a weak point of this review.
Additionally, almost all papers included an extensive set of references to existing protocols, that were not found by the methodology.
This could mean, that they are either too old, did not include the relevant search terms in the title or had only few citations.

Some articles even referenced other articles that were included in the set of reviewed papers.
In \cite{Zhu2019}, \citeauthor*{Zhu2019} outlined a number of security flaws with the protocol proposed in \cite{Gope2018},
including a lack of protection against desynchronization attacks and improvements on power efficiency.
In \cite{Gope2022}, \citeauthor*{Gope2022} critized the protocol proposed by \citeauthor*{Majzoobi2012} in 2012,
mentioning its lack of protection against replay and impersonation attacks.
This again suggests, that the set of selected papers is quite relevant.

The reviewed articles provided a diverse set of different interpretations on a challenge-response protocols,
with varying requirements, intended applications, focuses and security strengths or weaknesses.

The articles focus two main application areas: RFID and IoT. IoT is further categorized in general IoT applications
\ac{IoMT} and \ac{IoV} which consists of applications for traffic management, as well as \ac{V2G}.
The varying applications lead to a number of different underlying architectures.
Protocols focused on RFID are typically based on a three layer architecture, including the RFID tags, the readers and
a backend server \cite{Gope2018} \cite{Zhu2019}.
\cite{Gope2022} provided a unique approach, where devices talk to an IoT gateway, which is connected to an external service
provider over the internet for authentication.
\ac{V2G} also has an interesting architecture, using charging stations as a communication endpoint for the IoT devices,
which then talk to a backend server.
The wide variety of devices, which can be secured with PUF, like cars, medical devices and RFID tags, also highlights the
versatility of PUF technology.

Looking at the different types of PUF required by the protocols, it is surprising to see, that many protocols don't have
a specific requirement for a PUF implementation.
This suggests, that the PUF implementation is actually not as relevant to the protocols and infrastructure as previously assumed.
Some papers on the other hand include a requirement for strong PUFs \cite{Gope2018} \cite{Zhu2019}, while others
require a certain set of statistical properties \cite{Majzoobi2012} or a specific \ac{OPUF} implementation \cite{Gope2022}.
The loose requirements of many protocols do not make the choice of PUF irrelevant. There are generally strict statistical requirements for
PUFs used in authentication, which have been touched on in section \ref{sec:fuzzy_identification}.
Some protocols only provide support for ideal PUFs which are assumed to have perfect statistical properties,
but do not actually exist in the real world. Near perfect PUF robustness can be achieved through extensive
error correction using e.g. fuzzy extractors, but this is computationally expensive. \cite{Dodis2004}

On the verifier side, there are many approaches to managing the identity of provers.
\cite{Majzoobi2012} provides the use of a lightweight model of the PUF used to verify
responses. In \cite{Zhu2019}, only one \ac{CRP} is stored in the database at any time and a new
\ac{CRP} is agreed on between prover and verifier on each authentication run.
In \cite{Chatterjee2019}, a method is proposed, that does not require the verifier to store any CRPs.

All papers strongly focus on the security aspects of their protocols, as it is crucial in providing a
secure authentication system.
Table \ref{tbl:review_results} shows an in-depth overview of the security related features of each protocol.
For interpreting these results correctly, it is important to consider, that the security of each protocol
is interpreted without extensive knowledge in the field of cyber security and cryptography.
The table should therefore not be viewed as an actual representation of the security features of each protocol,
but rather, as an overview over which features were explicitly proven to be present.
As most papers used some kind of security framework for proofs, there is a high probability, that
some of these properties were implicitly proven in a non obvious way. These kinds of proofs could not
be taken into account, which means that some protocols might actually be more secure than the table suggests.

\begin{table}[H]
    \caption{Results of the Literature Review from a Security Perspective}
    \label{tbl:review_results}
    \begin{small}
        \begin{adjustwidth}{-0.5cm}{}
            \begin{center}
                \begin{tabular}{|l|c|c|c|c|c|c|c|c|c|}
                    \hline
                    Paper                                 & \cite{Majzoobi2012}         & \cite{Gope2018}             & \cite{Chatterjee2019}       & \cite{Braeken2018}          & \cite{Bansal2020}           & \cite{Zhu2019}              & \cite{Jiang2019}            & \cite{Gope2022}             & \cite{Hristea2019}          \\
                    \hline Section                        & \ref{sec:review_protocol_1} & \ref{sec:review_protocol_2} & \ref{sec:review_protocol_3} & \ref{sec:review_protocol_4} & \ref{sec:review_protocol_5} & \ref{sec:review_protocol_6} & \ref{sec:review_protocol_7} & \ref{sec:review_protocol_8} & \ref{sec:review_protocol_9} \\
                    \hline \hline Use Case                & Any                         & RFID                        & IoT                         & IoT                         & V2G                         & RFID                        & IoV                         & IoMT                        & RFID                        \\
                    \hline Required PUF                   & Any*                        & Strong                      & Strong                      & PUF-FSM                     & N                           & Any                         & Any                         & OPUF                        & Any                         \\
                    \hline Handles Noisy PUF resp.        & \checkmark                  & \checkmark                  & \checkmark                  & \checkmark                  &                             & \checkmark                  & \checkmark                  & \checkmark                  &                             \\
                    \hline \hline Machine Learning Attack & \checkmark                  &                             &                             &                             &                             & \checkmark                  &                             & \checkmark                  &                             \\
                    \hline Random Guessing Attack         & \checkmark                  & \checkmark                  & \checkmark                  &                             &                             & \checkmark                  &                             &                             &                             \\
                    \hline Desynchronization Attack       &                             & \checkmark                  &                             &                             &                             & \checkmark                  & ?                           & \checkmark                  & \checkmark                  \\
                    \hline Denial of Service Attack       &                             & \checkmark                  &                             & \checkmark                  &                             & \checkmark                  &                             & ?                           &                             \\
                    \hline Impersonation Attack           & \checkmark                  & \checkmark                  & \checkmark                  & \checkmark                  & \checkmark                  & \checkmark                  & \checkmark                  & ?                           & \checkmark                  \\
                    \hline Physical Attack                & \checkmark                  & \checkmark                  & \checkmark                  & \checkmark                  & \checkmark                  & \checkmark                  & \checkmark                  & \checkmark                  & \checkmark                  \\
                    \hline Cloning Attack                 & \checkmark                  & \checkmark                  & \checkmark                  & \checkmark                  & \checkmark                  & \checkmark                  & \checkmark                  & \checkmark                  & \checkmark                  \\
                    \hline Mutual Authentication          &                             & \checkmark                  &                             &                             & \checkmark                  & \checkmark                  & \checkmark                  & \checkmark                  & \checkmark                  \\
                    \hline Anonymity                      &                             &                             &                             &                             & \checkmark                  & \checkmark                  & \checkmark                  & \checkmark                  & \checkmark                  \\
                    \hline Forward Secrecy                &                             & \checkmark                  &                             &                             &                             & \checkmark                  &                             & \checkmark                  & \checkmark                  \\
                    \hline Scalable                       &                             & \checkmark                  &                             &                             &                             & \checkmark                  &                             & ?                           & \checkmark                  \\
                    \hline Security Proof                 & AM                          & AM                          & AE                          & A                           & AM                          & A                           & A                           & AS                          & A                           \\
                    \hline
                \end{tabular}
            \end{center}
        \end{adjustwidth}
        \begin{flushleft}
            \vspace{0.2em}
            * PUF must fit the strict PUF avalanche criterion \newline
            \checkmark = Proven to be secure against attack or implement feature \newline
            ? = Claimed but not proven \newline
            A = Argumentative, M = Mathematical, S = Simulation, E = Experiment, N = Not specified
        \end{flushleft}
    \end{small}
\end{table}

All papers proved the security of their protocol in some way. Methods used for this include
argumentative proofs with established frameworks like Session-Key Security or the Universal Composability Framework \cite{Chatterjee2019},
mathematical proofs showing the negligible chances of a successful attack through probabilistic calculations \cite{Gope2018}, experimental
proofs using actual hardware \cite{Chatterjee2019}, or the use of software simulations \cite{Gope2022}.
An important distinguishing feature between protocols is mutual authentication.
It means, that not only the entity proves their identity to the server, but also the other way around.
This is a central requirement for privacy and anonymity, as it ensures, that entities only have to share relevant identifying
information with a party they trust. This makes it useful for preventing location tracking in RFID systems.
In regards to security, the table shows a clear winner in terms of being able to prove the most security related features,
which is \cite{Zhu2019}, published by \citeauthor*{Zhu2019} in 2019.