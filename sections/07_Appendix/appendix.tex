\section{Guidelines for Literature Reviews by \citeauthor*{Brocke2009}}
\label{sec:appendix_brocke}

This is a summary of the guidelines for literature reviews in the information systems field,
proposed by \citeauthor*{Brocke2009} in \cite{Brocke2009}.
The methodology for the literature review in this thesis is loosely based on these five-steps.

\begin{enumerate}
      \item Analyze the scope and purpose of the review by using the taxonomy on
            literature reviews proposed in \cite{Cooper1988}. This acts as a basis for further
            steps and helps to keep the review focused on a specific goal
      \item Establish concepts about the topic, that are already widely known
            and recognized in the literature. This is best done using publications like existing review articles,
            textbooks or encyclopedias.
      \item After having gained a good understanding of the concepts and fundamentals of the topic,
            search terms should be devised, that can be used for the literature searching process.
            These search terms should then be applied in the following way:
            \begin{enumerate}
                  \item Find all journals, which publish articles related to the topic.
                  \item Find all databases, which include articles from those journals.
                  \item Query these databases using keywords and search terms devised from the previously
                        acquired knowledge.
                        Estimate the relevance of articles by analyzing title, abstract, number of citations and recency.
                        Include all relevant articles in the review.
                  \item Use backward and forward search to find additional relevant literature.
            \end{enumerate}
      \item Analyze all collected literature and synthesize new knowledge from it.
            This process is enabled by using a concept matrix (proposed in \cite{Webster2002}).
            Make sure to focus on the focus and goal specified in \emph{step one}.
      \item Create a research agenda that outlines, which questions have been answered by the
            existing research and which topics future research should focus on.
\end{enumerate}

\section{Exact Search Strings used in Literature Review}
\label{sec:appendix_search_strings}

\textbf{IEEE Xplore}
\begin{quote}
      ("Document Title":physical unclonable function OR "Document Title":puf OR "Document Title":physically unclonable function) AND ("Document Title":authentication OR "Document Title":challenge response OR "Document Title":challenge-response) AND ("Document Title":protocol)
\end{quote}

\textbf{ACM Digital Library}
\begin{quote}
      [[Title: physical unclonable function] OR [Title: puf] OR [Title: physically unclonable function]] AND [[Title: authentication] OR [Title: challenge response] OR [Title: challenge-response]] AND [Title: protocol]
\end{quote}

\textbf{Google Scholar}
\begin{quote}
      intitle:protocol AND (intitle:authentication OR intitle:challenge-response OR intitle:challenge OR intitle:response) AND (intitle:puf OR intitle:physically unclonable function OR intitle:physical unclonable function)
\end{quote}


\newpage
\section{Source Code for Implementation}
\label{sec:appendix_code}
\subsection{Pipfile}
This pipfile can be used to easily set up the appropriate environment using pipenv.
\lstinputlisting{implementation/Pipfile}

\subsection{support.py}
\lstinputlisting[language=python]{implementation/support.py}

\subsection{server.py}
\lstinputlisting[language=python]{implementation/server.py}

\subsection{tag.py}
\lstinputlisting[language=python]{implementation/tag.py}

\subsection{test\_auth.py}
\lstinputlisting[language=python]{implementation/test_auth.py}