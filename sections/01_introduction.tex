\section{Introduction}
\label{sec:introduction}

\subsection{Motivation}
\label{sec:motivation}

This thesis is part of a research project concerned with developing an authentication system
that makes use of \acp{PUF} for asserting identities.
Authentication is the act of verifying a person's or entity's identity through some identifying
feature, which proves, that they are, who they say they are. \cite[][p. 398]{Basavala2012}

In today's electronic access control systems, digital keys are used for authentication.
They are stored in databases and assigned to a physical entity like a person's
fingerprint or a smart card.
This approach can be problematic, as the digital key is not directly bound to
the physical object. If an attacker were to gain access to the key,
they could use it to create a copy of the physical entity
and gain access. \cite[][p. 81]{Gao2020}

A solution to this might be \acp{PUF}, which are physical devices, instead of
digitally stored keys.
During manufacturing, certain uncontrollable production tolerances lead to
each device differing slightly from every other.
This introduces unique characteristics and a certain variability and randomness into every \ac{PUF},
which can be used to create a unique fingerprint of every device without the need for a digital key.
While the variation is measurable, it is considered impractical to create an identical physical copy of
a \ac{PUF}, because it is deemed impossible to gain full control of the
manufacturing tolerances to imitate the behavior of another \ac{PUF} device. \cite[][p. 81]{Gao2020}

The physical characteristics of \acp{PUF} have to be measured to be able to infer an identity from them.
This is done using a \ac{CRP}. The challenge is some input given to the \ac{PUF}, which uses it to generate an output (the response),
leveraging its unique characteristics. For the same challenge, each individual \ac{PUF} will therefore generate a different response.
By previously generating and recording specific \acp{CRP} or using neural networks to evaluate a response to a
given challenge, the authenticity of a given \ac{PUF} can be established. \cite[][p. 81]{Gao2020}

The goal of the research project, that this thesis is part of, is to create an authentication system that can be used as an
electronic locking system for physical access protection of office buildings or other secure facilities.
The thesis focuses on the infrastructure behind the challenge-response based authentication system.
This includes specifying a protocol for communication between the components and developing a backend for generating
and sending challenges to the \ac{PUF} and verifying its responses.
The thesis is not concerned with the design of the \ac{PUF} modules themselves.

\subsection{Research Questions}
\label{sec:research_questions}

It is not possible to formulte specific research questions about the different components of the infrastructure from the beginning,
as it is not yet known at this point, what the infrastructure will look like.
However, looking at the overall project, this thesis aims to answer the following question, which is closely
related to the problem statement and title:

\begin{quote}
      \emph{RQ1: What does the infrastructure behind an authentication system using \acp{PUF} for physical access protection look like?}
\end{quote}

This question is answered by using literature research to understand the fundamentals behind these systems and
gaining knowledge about the current state of scientific research as a basis for the design.

As the implementation of a prototype is also part of the thesis, a second overarching question can be asked:

\begin{quote}
      \emph{RQ2: How can a functional prototype for the proposed infrastructure concept be implemented?}
\end{quote}

There is another aspect to this thesis, which is the evaluation of the proposed solution in the context of
existing concepts and research on the topic.
After answering these questions, the design and implementation should be evaluated to understand possible shortcomings
of the proposed solution and solidify the results.

\subsection{Methodology}
\label{sec:methodology}

To achieve the research goal of designing and implementing an infrastructure,
each of the sections are approached differently.

In the fundamentals section, qualitative literature research is used to gain insight into the current
state of scientific research and build a wide basis of knowledge that can be used for the conceptualization and implementation
of the infrastructure.
As there are is a broad range of topics to cover, no full systematic literature review is conducted in the fundamentals section.

The main sources for literature are be the EBSCO Discovery Service, IEEE Xplore, the ACM Digital Library, Springer Link and Google Scholar.
These search engines and databases cover the most important sources for scientific literature in the field.
The first step will be to gain a good understanding of \ac{PUF} related concepts.
From there, search terms are developed dynamically depending on the required knowledge.
Titles and abstracts of the most cited and relevant papers are scanned to find the papers
and books providing the most relevant information.
Additionally, if current review articles are found for a given topic,
snowballing might be used to find additional relevant information.

After building the fundamental knowledge basis, a systematic literature review of the most relevant scientific literature about
PUF-based authentication protocols is conducted.
The methodology for this review is based on the guidelines for conducting literature reviews in the information systems field
proposed by \citeauthor*{Brocke2009} in \cite{Brocke2009}. The exact approach is outlined in section \ref{sec:review_methodology}.

In the next step, the insights gained from the literature review are used to design and implement the protocol and supporting infrastructure.
Next, a concept is created based on the knowledge gained from literature research.
Existing solutions for each part of the infrastructure are evaluated based on the requirements of this thesis.

The goal for the implementation section of the thesis is to create a prototype.
Therefore, a language capable of rapid prototyping like Python is most likely
selected. Some components of the finished system, like the neural networks and \acp{PUF},
are not part of this thesis. Because of this, they are not going to be part of the prototype implementation and
will be simulated in some way. Additionally, tests are implemented to evaluate the finished prototype and to better understand
possible edge cases.

\subsection{Structure}
\label{sec:structure}

In section \ref{sec:fundamentals}, fundamental concepts of \ac{PUF} are explained.
First, this includes the definition of basic terms and description of basic concepts (\ref{sec:puf_def_of_terms}).
Statistical considerations of PUFs are examined (\ref{sec:statistical_considerations}) and the different types
of PUF are classified (\ref{sec:classification}). Next some examples of popular PUF implementations
are presented (\ref{sec:puf_implementations}). Concluding the fundamentals section,
central concepts of PUF-based authentication are examined (\ref{sec:puf_authentication}) and additional
applications for PUFs are presented (\ref{sec:puf_applications}).

In section \ref{sec:review}, a literature review of existing PUF-based authentication protocols is conducted.
It consists of specifying the methodology of the review process (\ref{sec:review_methodology}), presenting, which articles were chosen
for review (\ref{sec:review_collected_articles}), analyzing the collected literature (\ref{sec:review_analysis}) and a discussion of the results (\ref{sec:review_results}).

In section \ref{sec:implementation}, a prototype for an authentication protocol is implemented.
This includes the selection of a protocol (\ref{sec:imp_selection}), the design of the protocol (\ref{sec:imp_design}) and a presentation of
the proposed solution (\ref{sec:imp_solution}).

Finally, the thesis is concluded (\ref{sec:conclusion}) with an evaluation and results and an outlook on future research topics.