%
%
%
%
%
%
%
% Quelle: https://www.namsu.de/Extra/pakete/Listofsymbols.pdf
% Wie ind er Quelle beschrieben führt das Verwenden von Umlauten oder ß zu einem Fehler.
% Hier werden die Symbole definiert in folgender Form:
% \newsym[Beschreibung]{Symbolbefehl}{Symbol}
\opensymdef
\newsym[Aufrechter Buchstabe]{AB}{\text{A}}
\newsym[Menge aller natuerlichen Zahlen ohne die Null]{symnz}{\mathbb{N}}
\newsym[Menge aller natuerlichen Zahlen einschliesslich Null]{symnzmn}{\mathbb{N}_{0}}
\newsym[Menge aller ganzen Zahlen]{GZ}{\mathbb{Z}}
\newsym[Menge aller rationalen Zahlen]{RatZ}{\mathbb{Q}}
\newsym[Menge aller reellen Zahlen]{RZ}{\mathbb{R}}
\closesymdef
